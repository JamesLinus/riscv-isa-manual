\chapter{Hypervisor Extension, Version 0.0}
\label{hypervisor}

This chapter describes the RISC-V hypervisor extension, which virtualizes
the supervisor-level architecture to efficiently host guest operating
systems atop a type-1 or type-2 hypervisor.  The hypervisor extension adds
a second level of address translation and protection to virtualize the
memory and memory-mapped I/O subsystems.  The extension also replaces
S-mode with two new privilege modes: {\em hypervisor mode} (H-mode), where
the hypervisor runs, and {\em virtual supervisor} mode (VS-mode), where the
guest runs.  H-mode acts the same as S-mode, but with additional
instructions and CSRs that support hosting a VS-mode guest.  VS-mode
appears to act the same as S-mode.  S-mode operating systems can execute
without modification in either H-mode or VS-mode.

An H-mode OS or hypervisor interacts with the machine through the same SBI
as would an S-mode OS.  An H-mode hypervisor is expected to implement the
SBI for a VS-mode guest.

\begin{commentary}
The privileged architecture is designed to simplify the use of classic
virtualization techniques, where a guest OS is run at user-level, as
the few privileged instructions can be easily detected and trapped.
The hypervisor extension improves virtualization performance by
reducing the frequency of these traps.
\end{commentary}

\section{Privilege Modes}

The current {\em virtualization mode}, denoted V, indicates whether
the hart is currently executing in
a VS-mode guest or a U-mode program on a VS-mode guest.
When V=1, the hart is either in VS-mode or in U-mode under a VS-mode OS.
When V=0, the hart is either in M-mode or H-mode, or in U-mode under an
H-mode OS.  The virtualization mode also indicates whether two-level
address translation is active.  Table~\ref{h-operating-modes} lists the
possible operating modes of a RISC-V hart with the hypervisor extension.

\begin{table*}[h!]
\begin{center}
\begin{tabular}{|c|c||l|l|l|}
  \hline
   Privilege & Virtualization & \multirow{2}{*}{Abbreviation} & \multirow{2}{*}{Name} & Two-Level \\
   Mode      & Mode (V)       &                               &                       & Translation \\ \hline
   0         & 0              & U-mode  & User mode & Off \\
   1         & 0              & H-mode  & Hypervisor mode & Off \\
   3         & 0              & M-mode  & Machine mode & Off \\
   0         & 1              & U-mode  & User mode & On \\
   1         & 1              & VS-mode & Virtual-supervisor mode & On \\
  \hline
 \end{tabular}
\end{center}
\caption{Operating modes with the hypervisor extension.}
\label{h-operating-modes}
\end{table*}

\section{Hypervisor CSRs}

An H-mode hypervisor uses the S-mode CSRs to interact with the exception,
interrupt, and address-translation subsystems, as would an S-mode OS.
Additional H-mode CSRs---{\tt hstval}, {\tt hsstatus}, {\tt hsedeleg}, and
{\tt hsideleg}--control the behavior of a VS-mode guest.

Additionally, several {\em alternate-supervisor} CSRs, accessible to H-mode,
are copies of a corresponding supervisor CSR.  When transitioning between
virtualization modes (V=0 to V=1, or vice-versa), the implementation swaps the
alternate-supervisor CSRs with their supervisor counterparts.  When V=0, the
alternate-supervisor CSRs contain VS-mode's verison of those CSRs, and the
supervisor CSRs contain H-mode's version.  When V=1, the alternate-supervisor
CSRs contain H-mode's version, and the supervisor CSRs contain VS-mode's
version.




%Kill aedeleg, acounteren, medelegv
%
%ECALL-from-VS = cause 10
%
%call it Hypervisor Extensions
%
%a* regs? “as” alternate super
%stvalv = hstvalv?
%vm* = “hs” h-extended supervisor
%
%Describe different VM use cases:
%TVM/TSRET-style
